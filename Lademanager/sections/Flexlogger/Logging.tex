\subsection{Logging Allgemein}
% Big Data in der Praxis - Lösungen mit Hadoop, Spark, HBase und Hive. Daten speichern, aufbereiten, visualisieren. 2. erweiterte Auflage, von  Jonas Freiknecht, Stefan Papp, 2018, S
% https://www.omega.com/en-us/resources/data-loggers
Data Logging kann dazu verwendet werden, Code zu debuggen oder große Datenmengen zu speichern, um diese später auszuwerten. Ein Datenlogger ist im Allgemeinen ein Instrument, welches Veränderungen unter bestimmten Bedingungen während einer gewissen Zeitspanne aufzeichnet. Ein Datenlogger verwendet meist Sensoren, um Daten zu sammeln. Anschließend können die Daten ausgelesen, visuell dargestellt und/oder ausgewertet werden. Die Daten, welche von Datenloggern ausglesen werden, betragen meist, Druck, Temperatur, Luftfeuchtigkeit, Spannung oder Stromstärke. 

Die gespeicherten (geloggten) Daten können dazu verwendet werden, 

\begin{compactitem}
    \item um die Temperatur und die Luftfeuchtigkeit in einem Gebäude zu überprüfen.
    \item Information zur Gebäudewartung zur Verfügung zu stellen, dies betrifft das Heizen, die Belüftung, die Klimatisierung. Diese ständige Überprüfung der Daten kann den Energieverbrauch reduzieren.
    \item die Wachs-Bedingungen von Pflanzen in der Landwirtschaft zu überwachen. 
    \item die Impfstoff-Lagerung in medizinischen Einrichtungen zu überwachen. 
    \item die Temperatur von Lebensmittel zu überprüfen.
\end{compactitem}


\subsection{Log4J}
% https://books.google.at/books?hl=de&lr=&id=hZBimlxiyAcC&oi=fnd&pg=PA9&dq=log4j&ots=QiOna081Z6&sig=9h3lwPqN-rm07kAHln-tLSUfgJY&redir_esc=y#v=onepage&q=log4j&f=false The Complete Log4j Manual, von: Ceki Gülcü S15
Framework, um Anwendungsmeldungen von JAVA zu loggen.
Das Design von Log4J konzentriert sich vor allem darauf, schnell, flexibel und leicht verständlich zu sein. 

