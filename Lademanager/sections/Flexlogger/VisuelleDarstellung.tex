\subsection{Angular}
% https://angular.io/ 

Angular ist eine Plattform, um Web-Applikationen zu erstellen, welche für die Desktop- sowie für die mobile Anwendung gleichfalls funktionieren sollen. Gebaut wurde Angular in der Programmiersprache TypeScript und es inkludiert folgende Fähigkeiten(?):

\begin{compactitem}
    \item Ein Komponenten-basiertes Framework, um skalierbare Web-Applikation zu erstellen.
    \item Eine Sammlung von Bibliotheken, welche eine große Varietät von Features beinhaltet, Beispiele dafür sind: Routing, das Management von Formularen, sowie eine Client-Server Kommunikation. Diese Bibliotheken sind laut Angular gut in die Plattform eingebunden.       
    \item Eine Auswahl von Entwickler-Tools, welche hilfreich sind, um den Code zu entwickeln, zu testen, zu bauen und upzudaten.
\end{compactitem}

\subsubsection{Komponenten}
Komponenten sind die Baublöcke, um eine Applikation zusammenzustellen. In einer Komponente sind folgende Segmente: 

\begin{compactitem}
    \item ein HTML-Template
    \item ein CSS-Style Template      
    \item einem @Component-Part, in welchem folgende Informationen definiert werden:
    \subitem Ein CSS-Selektor, welcher definiert wie die Komponente in einem Template verwendet wird. Dieser Selektor kann anschließend in ein HTML-File eingebunden werden. Passiert dies, wird dieser Selektor eine Instanz der Komponente de HTML-Files.
    \subitem Ein HTML-Template welches Angular anleitet, wie es diese Komponente zu rendern hat.
    \subitem Ein optionales Set von CSS-Styles, welche das Aussehen des HTML-Elements definiert.
\end{compactitem}

\begin{lstlisting}[language=java,caption=Beipsiel für eine minimierte Angular Komponente,label=lst:impl:foo]
    import { Component } from '@angular/core';

    @Component({
      selector: 'hello-world',
      template: `
        <h2>Hello World</h2>
        <p>This is my first component!</p>
      `
    })
    export class HelloWorldComponent {
      // The code in this class drives the component's behavior.
    }
\end{lstlisting}

\subsection{CanvasJS}
