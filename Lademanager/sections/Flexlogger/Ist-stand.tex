Die Firma FlexSolution befindet sich in der Lage, die benötigten Daten zu loggen (abzuspeichern) mithilfe von Log4J. Allerdings werden dabei die Daten zuerst in einer Datei gespeichert, darauffolgend komprimiert und in einem Ordner abgelegt. Nachdem diese Daten komprimiert wurden, gibt es wenig performante Möglichkeiten, die Daten auszulesen, da diese nicht zentral abgespeichert werden und ein Zugriff auf die Daten sich somit als umständlich erweist. Das Ziel dieses Teils der Diplomarbeit ist, die Daten in Echtzeit zu loggen und in einer zusätzlichen Datenbank abzuspeichern, um diese anschließend grafisch ansprechend darzustellen.
 
