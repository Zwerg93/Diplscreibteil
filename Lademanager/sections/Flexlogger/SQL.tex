\subsection{Allgemein}
% Einführung in SQL- Daten erzeugen, bearbeiten und abfragen von: Alan Beaulieu p15-27
% https://edufs.edu.htl-leonding.ac.at/moodle/pluginfile.php/102732/mod_resource/content/1/PLSQL-Skript%20v04.pdf
SQL ist eine Programmiersprache, welche entwickelt wurde, um die Daten in einer relationalen Datenbank zu bearbeiten. Außerdem kann damit die Struktur der Datenbank abgeändert werden.
 
Es gibt verschiedenste SQL-Befehlsgruppen, welche jeweils einen anderen Zweck erfüllen.
 
\begin{center}
    \begin{tabular}{ |c|c|c|c| }
     \hline
     \multicolumn{3}{|c|}{SQL-Befehlsgruppen } \\
     \hline
     \hline
     Database Manipulation (DML) & Database Definition (DDL) & Database Operation  \\
     \hline
     \hline
     DELETE & CREATE & SERVERERROR \\
     \hline
     INSERT & ALTER & LOGON/LOGOFF \\
     \hline
     UPDATE & DROP & STARTUP/SHUTDOWN \\
     \hline
    \end{tabular}
    \end{center}
 
Mithilfe all dieser verschiedenen Gruppen ist es möglich, eine Datenbank zu erstellen, zu verwalten und zu löschen.
 
Um beispielsweise eine Tabelle in einer Datenbank zu erstellen, verwendet man das Schlüsselwort \texttt{CREATE}, welches in der Data Definition Language gefunden werden kann. \ref{lst:impl:createTable}
 
\begin{lstlisting}[language=sql,caption=CREATE table,label=lst:impl:createTable]
    CREATE TABLE animal (
        animalName VARCHAR(20),
        animalAge NUMBER(3)
        );
\end{lstlisting}
 
Um nun eine Zeile in die Tabelle hinzuzufügen, wird ein INSERT-Statement benötigt. \ref{lst:impl:insertTable}
 
\begin{lstlisting}[language=sql,caption=CREATE table,label=lst:impl:insertTable]
        INSERT INTO animal (animalName, animalAge) VALUES ('snake', 3);
\end{lstlisting}
 
Ein anderer wichtiger Befehl in SQL ist der SELECT-Befehl. Mit ihm kann aus jeder beliebigen Tabelle ein bestimmter Teil ausgegeben werden. Dabei können verschiedenste Kriterien übergeben werden, wie z.B. dass die auszugebenden Daten einen bestimmten Wert haben müssen.
 
\subsection{PLSQL}
 
\subsection{Kommunikation mit Datenbanken}
% Einführung in SQL - Daten erzeugen, bearbeiten und abfragen von: Alan Beaulieu p29-47
Um SQL in einer Datenbank zu verwenden, muss dazu aller erst eine Datenbank angelegt werden. Dabei gibt es verschiedenste Datenbanken, welche man je nach Performance, bzw. benötigter Größe auswählen sollte. Siehe dazu Kapitel 4.6.
 
 
