\begin{spacing}{1}
    \chapter*{Abstract}
\end{spacing}
\begin{wrapfigure}{r}{0.3\textwidth}
    \begin{center}
      \includegraphics[width=0.2\textwidth]{pics/Firmenlogo.jpeg}
    \end{center}
\end{wrapfigure}
The work 'Elektroauto Lademanager' is about a project that was divided into two parts.

The first part consists of a web interface developed for the company Flexsolution. This required a way to charge the new electric cars. To enable this, new charging stations were installed at the beginning of this work. In the interface described above, the data of the charging station is displayed. The app also allows you to control the charging station.

The second part is about a logger that collects, processes, stores and finally displays internal data on a website. The data collected are usually values of machines or sensors that are installed in the company. These data were stored before but could not be read easily or tracked. To enable an easy way of monitoring this data, this part of the project was launched.
\newpage
\begin{spacing}{1}
    \chapter*{Zusammenfassung}
\end{spacing}
\begin{wrapfigure}{r}{0.3\textwidth}
    \begin{center}
      \includegraphics[width=0.2\textwidth]{pics/Firmenlogo.jpeg}
    \end{center}
\end{wrapfigure}

Die Arbeit 'Elektroauto Lademanager' handelt von einem Projekt, welches in zwei Teile aufgeteilt wurde.
Der erste Teil besteht aus einem Webinterface, welches für die Firma Flexsolution entwickelt wurde. Diese benötigte eine Möglichkeit, um die neuen E-Autos aufzuladen. Um dies zu ermöglichen, wurden im Unfang dieser Arbeit neue Ladestationen angebracht. In dem zuvor beschriebenen Interface werden die Daten der Ladestation angezeigt. Außerdem ermöglicht die App die Ladestation zu steuern.
Der zweite Teil handelt über einen Logger, welcher firmeninterne Daten sammelt, aufbereitet, abspeichert und schließlich auf einer Website anzeiget. Die gesammelten Daten sind meist Werte von Maschinen oder Sensoren, welche in der Firma installiert sind. Diese Daten wurden zuvor zwar abgespeichert aber konnten nicht leicht abgespeichert oder nachverfolgt werden. Um eine Überwachung dieser Daten zu ermöglichen, wurde dieser Teil des Projektes ins Leben gerufen.
