Die Datenpunkte sind das Herzstück der Datenübertragung. Sie ermöglichen es, einfache Daten oder komplexe Objekte innerhalb der Firma zu übertragen. Die einzelenen Felxtask sind so ausgelegt, dass alle untereinander kommunizieren können. Somit ermöglicht dieses System einen schnellen, unkomplizierten Austausch jeglichder Daten. In diesem Kapitel wird beschriebem wie die Datenpunkte aufgebaut sind, wie sie angelegt werden und wie man sie benützen kann.

\subsection{Struktur eines Datenpunktes}\label{Datenpunkte} \setauthor{Pouget Marcel}

Die Datenpunkte sind im Grunde alle gleich aufgebaut. Ein Datenpunkt (dp) besteht immer aus einer:

\begin{compactitem}
    \item ID: kann nach Belieben benannt werden. Meist ein String, um den Nutzen zu veranschaulichen und die DP´s voneinander unterscheiden zu können. Die ID muss für jedes Device eindeutig sein.
    \item device\_id: gibt an, zu welchem Device der Datenpunt gehört. Ein Device kann in dem config.db-File hinzugefügt werden. Das dient dazu, um mehrere Datenpunkte mit derselben ID für mehrere Devices verwenden zu können. Die device\_id besteht immer aus einem INT. Um Datenpunkte nutzen zu können, muss mindestens ein Device angelegt werden. Die ID wird automatisch immer um eins erhöht. Im Normalfall ist schon ein Device mit der ID 1 und dem Label dev1 angelegt.
    \item Label: Hier kann man den Datenpunkt kurz beschreiben, um seine Funktion zu dokumentieren. Das wird dafür verwendet, um den Entwicklern das Verwalten der Datenpunkte zu vereinfachen.
    \item SpecificAddress: Ist meistens derselbe String wie die ID, kann jedoch auch etwas anderes sein. Wird dazu benötigt, um den Datenpunkt im Task anzulegen.
    \item SpecificDatatype: Hier kann ein String mitgegeben werden, um in einem Task einen Datentyp für den Datenpunkt festzulegen, bzw. zwischen Gruppen von Dp zu unterscheiden. Im Task für das Gateway wird der specificDatatype dafür genutzt, um zwischen Datenpunkten, die nur für Modbus zuständig sind, und den anderen Datenpunkten zu unterscheiden.
    \item Datatype: hier kann man angeben, welchen Datentyp die Value haben soll. Man kann hier zwischen 'INT', 'REAL', 'BOOL' und 'CMD' wählen.
    \item Intervall: Hier kann man ein Intervall angeben, mit welchem der Datenpunkt in die Flexcloud gepusht werden soll. Standardmäßig ist –1 eingestellt, was bedeutet, dass der Datenpunkt nur nach Anfrage des Tasks bzw. bei Änderung der Value gepublished wird.
    \item Value: Hier werden die Werte eines Datenpunktes gespeichert. Man kann in der conf.db Datei einen Standardwert angeben, doch meistens werden diese Werte in einem Task gesetzt bzw. ausgelesen.
\end{compactitem}

\subsection{Anlegen eines Datenpunktes} \setauthor{Pouget Marcel}


Um einen neuen Datenpunk in einem Task anzulegen, muss man sich zuerst auf die VM (Virtual machine),  auf welcher der Flextask sich befindet, verbinden. In dem Verzeichnis mit der conf.db Datei muss man mit SQLITE3 die Datei öffnen. Dort kann man dann mit einem insert bzw. Update Statement einen neuen Datenpunkt hinzufügen, aktualisieren oder löschen.

\begin{lstlisting}[language=sql,caption=SQL Example,label=lst:impl:foo]
    INSERT INTO datapoint VALUES ('Wallbox_1_startCharching_icon_state',1,'' ,'state_[Wallbox_1_startCharching_icon]','','','INT',-1,-1,0.0,'');
\end{lstlisting}

\subsection{Nutzung eines Datenpunktes} \setauthor{Pouget Marcel}

Somit ist der Datenpunkt in der Datenbank angelegt und ready to use. Um nun in einem Flextask auf so einen Datenpunkt Zugriff zu haben, muss man ihn erstmal anlegen und die 'SPECIFIC\_ADDRESS' angeben:

\begin{lstlisting}[language=java,caption=Alengen eines Datenpunktes,label=lst:impl:foo]
    Datapoint datapoint = StaticData.datapoints.getDatapoint(Datapoints.DatapointField.SPECIFIC_ADDRESS, "SPECIFIC_ADDRESS_DES_DATENPUNKTES"); 
\end{lstlisting}

Nun hat man mehrere Möglichkeiten, wie man Daten aus dem Datenpunkt lesen kann:
\begin{compactenum}
    \item Man ruft die Methode 'setDatapointChangedCommand()' auf. Diese Methode überschreibt eine in der Klasse Flextask, und ist dafür da, um auf Änderungen des Datenpunktes zu hören.
    
    \begin{lstlisting}[language=java,caption=Example datapoint usage,label=lst:impl:foo]
        datapoint.setDatapointChangedCommand(new DatapointCommand() { 
            @Override 
            public DatapointResponse execute(DatapointResult result) { 
         
               System.out.println(datapoint.getValue()); 
                return new DatapointResponse(DatapointResponse.ResponseCode.OK); 
            } 
        }); 
    \end{lstlisting}
    Diese Methode gibt bei jeder Änderung des Wertes den neuen Wert in der Konsole aus. Hier kann man den Wert speichern, andere Methoden ausführen, oder sonst alles machen, was man in einer Methode machen könnte.

    \item Man kann jederzeit im Code 'datapoint.getValue()' aufrufen. Mit dieser Methode bekommt man immer den jetzigen Wert des Datenpunkts zurück.
    \item Wie schon in den vorhergegangenen Kapiteln erwähnt, kann man über alle Datenpunkte jedes Device iterieren, und je nach Bedingung ein DatapointCahngedCommand anhängen. Dies bewirkt, dass eine Methode immer dann aufgerufen wird, wenn sich einer der Datenpunkte ändert. Mit der ID der einzelnen Devices kann man dann auf die SpecificAddress zugreifen, und die einzelnen Fälle mit einem switch Statement abdecken.
    \begin{lstlisting}[language=java,caption=Example multible datapoint usage,label=lst:impl:foo]
        for (Device dev : StaticData.devices.getDevices()) { 
    for (Datapoint dp : StaticData.datapoints.getDatapoints(dev.getId())) { 
        if (!dp.getSpecificDataType().isEmpty()) { 
            dp.setDatapointChangedCommand(dpCmd);                 //!!! 
        } 
    } 
} 

 

DatapointCommand dpCmd = new DatapointCommand() { 
      @Override 
    public DatapointResponse execute(DatapointResult datapointResult) { 
 
        int deviceID= datapointResult.getDeviceId();  
        switch (datapointResult.getDpId()) { 
            case "datapoint_example_id": 
                System.out.println(datapointResult.getValue() + " " + deviceID ); 
                 break; 
                    } 
        return new DatapointResponse(DatapointResponse.ResponseCode.OK); 
    } 
}; 

    \end{lstlisting}
    In diesem Beispiel werden die DeviceID und der aktuelle Wert des Datenpunktes, welcher sich geändert hat, ausgegeben.
\end{compactenum}

Mit 'datapoint.setValue(value)' kann man einem Datenpunkt einen neuen Wert zuweisen. Dieser wird dann automatisch in die FLexcloud gepublished, und kann von einem anderen Task abgefangen werden.


\subsection{Datapoint mapping} \setauthor{Pouget Marcel}

Um zwei Datenpunkte von zwei verschiedenen Tasks zu mappen, und somit Daten zu übertragen, muss man einen Eintrag in der 'datapoint\_map' hinzufügen. Um die Daten zu empfangen, muss man die conf.db des zweiten Tasks aktualisieren. Zu beachten ist, dass das Mapping nur bei dem Task gemacht werden muss, welcher die Daten bekommen soll. Man verbindet also den Datenpunkt eines anderen Tasks auf einen eigenen Datenpunkt. Wichtig dabei ist, dass man bei der 'datapoint1\_id' zuerst das Label des Devices, zu welchem der Datenpunkt gehört, und anschließend die ID des ersten Datenpunkt angibt. 'atapoint2\_id' wird zuerst der Name des anderen Tasks angegeben, dann das Label des Devices, und anschließend die ID des Datenpunktes.



Beispiel: Ich möchte von dem Flextask 'modbuswallbox', welcher in dem Projekt als das Gateway bezeichnet wurde, den Wert der Ladedauer der ersten Wallbox zum 'chargecontroller' schicken. In dem Task 'modbuswallbox' muss dafür ein Device mit dem Label 'wb1' und ein Datenpunkt mit der ID 'ladedauer\_wb1' angelegt werden. Wird nun im Code auf diesen Datenpunkt was gepublished, sieht man 'modbuswallbox.wb1.ladedauer\_wb1'. Damit die Daten jetzt auch ankommen, muss im Task für den Charge-controller ein Datenpunkt mit der ID 'ladedauer\_wb' angelegt werden. In die 'datapoint\_map' Tabelle muss dann bei der 'datapoint1\_id' das Device und der Datenpunkt angegeben werden, und bei der 'datapoint2\_id' zuerst der Task Name, das Device und dann der Datenpunkt, der die Daten pulished.

\begin{lstlisting}[language=sql,caption=Eintrag in die datapoint\_map Tabelle,label=lst:impl:foo]
    dev1.ladedauer_wb                       modbuswallbox.wb1.ladedauer_wb1
\end{lstlisting}