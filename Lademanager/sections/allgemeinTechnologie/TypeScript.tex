% < https://www.typescriptlang.org/
% < JavaScript für Java-Entwickler Autor: Oliver Zeigermann 2015 S117-132
TypeScript ist eine Weiterentwicklung von Javascript, welche einige Vorteile mit sich bringt. TypeScript fügt zum Beispiel zusätzliche Syntax zu JavaScript hinzu, um eine engere Integration mit dem Editor der Wahl zu ermöglichen. Somit können Errors früher erkannt werden. TypeScript Code wird zu JavaScript Code umgewandelt und als dieser ausgeführt. Somit läuft es überall, wo auch JavaScript läuft: Dies beinhaltet einen beliebigen Browser, Node.js oder in einer App. Zusätzlich kann JavaScript auch in TypeScript verwendet werden. Ein weiterer großer Vorteil von TypeScript ist, dass es eine automatische Typendeklaration besitzt. Das heißt, dass bei der Erstellung einer Variable kein Typ mitangegeben werden muss, sondern dieser beim Kompilieren des Programms (aufgrund des Inhalts der Variable) selbst zugewiesen wird. \ref{lst:impl:automatischeZuweisung}
\cite{JavaScriptJavaEntwickler} 
\cite{TypeScriptOverview} 

\begin{lstlisting}[language=java,caption=TypeScript automatische Zuweisung,label=lst:impl:automatischeZuweisung]
    // Der Type String wird hier bei der Kompilation automatisch zugewiesen
    let helloWorld = "Hello World";
\end{lstlisting}