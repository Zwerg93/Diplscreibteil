Java ist eine kostenlose objektorientierte Programmiersprache, welche im Jahr 1995 von dem Unternehmen Sun Microsystems veröffentlicht wurde. Das Unternehmen Oracle kaufte im Jahr 2010 das Unternehmen und entwickelt seitdem die Sprache kontinuierlich weiter. Die Sprache ist neben C eine der bekanntesten Programmiersprachen weltweit. Sie kann auf allen bekannten Systemen installiert werden und garantiert durch den Aufbau, dass es auf jedem System lauffähig ist. Um das zu ermöglichen, setzt Java auf eine virtuelle Laufzeitumgebung (Java VM), auf welcher der Code ausgeführt wird. Das Java Development Kit (JDK) ist dabei ein wichtiger Bestandteil, denn der integrierte Compiler übersetzt den menschlichen Code in Maschinencode.

Java ermöglicht es bei der Entwicklung große Teile des Codes wieder zu verwenden. Das Ganze ist in Klassen und Objekte eingeteilt. Vergleichbar ist es mit einer Bauanleitung. So kann eine Klasse z.B. eine Anleitung für ein Auto sein mit Attributen wie Rädern, PS, Farbe und einem Baujahr. Sollten nun mehrere Autos gebraucht werden, können Objekte vom Typ Auto erstellt werden. Dabei hat jedes Auto dieselben Variablen, aber unterschiedliche Werte. Es können auch Funktionen aus der Klasse benutzt werden.

Dies bietet die Möglichkeit, eine Vielzahl von Code zu recyclen. Ein weiterer wichtiger Punkt in Java sind die sogenannten Build Tools, die es erlauben, Librarys und Dependencies zu dem Code hinzuzufügen. Ein Beispiel dafür wäre das Framework Quarkus, welches den Einsatz mit Rest-Endpunkten erleichtert.


\cite{javaOracle}
\cite{javaJava}
\cite{javaHegelit}
\cite{javaWikipedia}


Eine Typische erste Java-Anwendung ist Hello World,

\begin{lstlisting}[language=HTML,caption=Hello World,label=lst:impl:HelloWorld]
    public class TheMain {
    public static void main(String[] args) {
        System.out.println("Hello World!");
    }
}
  \end{lstlisting}