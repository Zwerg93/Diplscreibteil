HTML, CSS und JavaScript sind allesamt Komponenten, um eine WebSeite zu bauen. JavaScript ist dabei für die Funktionalität zuständig, CSS für das Design und HTML für den Grundaufbau.

Diese Sprachen wurden verwendet, um den zweiten Teil der Diplomarbeit, die visuelle Darstellung der Daten auf einer Website, zu realisieren.

\subsection{HTML}
% < Durchstarten mit HTML5 Autor: von: Mark Pilgrim, 2011 S. 18-23
HTML gilt als Grundgerüst einer Website. HTML besitzt sogenannte Tags, welche je über einen eigenen Nutzen verfügen. Durch diese Tags kann der Titel einer Website festgelegt werden, verschiedenste Bilder, Texte und anderer Content hinzugefügt werden. Außerdem bietet HTML die Möglichkeit, einer Webseite die passende Struktur zu verleihen. \cite{durchstartenHTML}

Um das Strukturieren einer Website zu vereinfachen, gibt es fest zugewiesene Tags. Um die wichtigsten kurz aufzulisten:
 
\begin{compactitem}
    \item [<title></title>]
    \item Definiert den Titel eines HTML Dokuments
    \item [<div></div>, <nav></nav>]
    \item Kennzeichnen eines Abschnitts bzw. einer Navigationsleiste in einem HTML Dokument
    \item [<p></p> <h1></h1>]
    \item Verantwortlich, dem HTML Dokument Text (p) und Überschriften (h) hinzuzufügen
    \item [<img> <video> <audio>]
    \item Mithilfe dieser Tags kann ein Bild/Video/Audio eingefügt werden
\end{compactitem}\cite{durchstartenHTML}

Ein Beispiel für ein HTML-File kann im nächsten Abschnitt gefunden werden. \ref{lst:impl:htmlJavaScript}
 
\subsection{CSS}
Mithilfe von CSS können Farben, Formen, Abstände und andere Designmittel einer Webseite geändert werden. CSS geht Hand in Hand mit HTML und kann jede Klasse oder auch jeden Tag, der in einem HTML-File vorkommt, ansprechen. Ein paar der wichtigsten Attribute, welche man mit CSS ansprechen und folglich verändern kann, sind: color (Farbe), width und height (Breite und Höhe eines Elements), sowie margin und padding (Abstand eines Elements zu anderen Elementen).
 
CSS kann wie JavaScript entweder direkt im HTML eingebettet \ref{lst:impl:cssEinbettung}, oder als externes File angegeben werden. Dies dient wiederum der Übersichtlichkeit und sollte je nach Größe des Projekts entschieden werden.
\cite{durchstartenHTML}

\begin{lstlisting}[language=java,caption=CSS Einbettung,label=lst:impl:cssEinbettung]
    <link rel="stylesheet" href="style.css" type="type/css" />
\end{lstlisting}
 
\subsection{JavaScript}
% < https://www.typescriptlang.org/docs/handbook/typescript-from-scratch.html
% < JavaScript für Java-Entwickler Autor: Oliver Zeigermann 2015 S9-12
JavaScript wurde ursprünglich dafür entwickelt, um als einfache Skriptsprache für kurze Code-Snippets im Browser eingesetzt zu werden. Nach einiger Zeit wurde es allerdings populärer und wurde immer öfters und für längere Codestellen eingesetzt. Webbrowser reagierten darauf sehr positiv und optimierten die Ausführung von JavaScript. Nach dem Erfolg, den JavaScript bei Browsern feierte, wurde es unter anderem auch für node.js und JS-Server verwendet.
\cite{typeJavaScript}
JavaScript kann verwendet werden, indem es in ein HTML-Dokument eingebettet wird \ref{lst:impl:htmlJavaScript}, allerdings ist es ab einer bestimmten Größe des Programms empfehlenswert, HTML und JavaScript in verschiedene Files zu verlagern, um die Lesbarkeit zu unterstützen.
\cite{JavaScriptJavaEntwickler} 

\begin{lstlisting}[language=java,caption=HTML mit eingebettetem JavaScript,label=lst:impl:htmlJavaScript]
    <!DOCTYPE html>
    <html>
        <head>
            <title>Hello World!</title>
        </head>
        <script>
            alert('Hello World!'); // wenn die Seite aufgerufen wird, wird mithilfe von JavaScript ein Alert-Fenster mit 'Hello World!' ausgegeben
        </script>
        <body>
        </body>
    </html>
\end{lstlisting}