Um davor aufbereitete Daten sinnvoll visuell darzustellen, ist es wichtig, die richtigen Plattformen beziehungsweise Frameworks zu verwenden. In folgendem Kapitel werden verwendete Plattformen und Frameworks erklärt. 
\subsection{Angular}
% < 1 https://angular.io/
 
Angular ist eine Plattform, um Web-Applikationen zu erstellen, welche für die Desktop- sowie für die mobile Anwendung gleichfalls funktionieren sollen. Gebaut wurde Angular in der Programmiersprache TypeScript und es inkludiert folgende Fähigkeiten:
 
\begin{compactitem}
    \item Ein Komponenten-basiertes Framework, um eine skalierbare Web-Applikation zu erstellen.
    \item Eine Sammlung von Bibliotheken, welche eine große Varietät von Features beinhaltet, Beispiele dafür sind Routing, das Management von Formularen sowie eine Client-Server-Kommunikation. Diese Bibliotheken sind laut Angular gut in die Plattform eingebunden.
    \item Eine Auswahl von Entwickler-Tools, welche hilfreich sind, um den Code zu entwickeln, zu testen, zu bauen und upzudaten.
\end{compactitem}
 
\subsubsection{Komponenten}
Komponenten sind die Baublöcke, um eine Applikation zusammenzustellen. In einer Komponente sind folgende Segmente:
 
\begin{compactitem}
    \item ein HTML-Template
    \item ein CSS-Style Template
    \item einem @Component-Part, in welchem folgende Informationen definiert werden:
    \subitem Ein CSS-Selektor, welcher definiert, wie die Komponente in einem Template verwendet wird. Dieser Selektor kann anschließend in ein HTML-File eingebunden werden. Passiert dies, wird dieser Selektor eine Instanz der Komponente des HTML-Files.
    \subitem Ein HTML-Template, welches Angular anleitet, wie es diese Komponente zu rendern hat.
    \subitem Ein optionales Set von CSS-Styles, welche das Aussehen des HTML-Elements definieren.
\end{compactitem}
\cite{angularOfficialSite}
 
\begin{lstlisting}[language=java,caption=Beispiel für eine minimierte Angular Komponente,label=lst:impl:angularBsp]
    import { Component } from '@angular/core';
 
    @Component({
      selector: 'hello-world',
      template: `
        <h2>Hello World</h2>
        <p>This is my first component!</p>
      `
    })
    export class HelloWorldComponent {
      // The code in this class drives the component's behavior.
    }
\end{lstlisting}
\cite{angularOfficialSite}
 
\subsubsection{Anbindung einer Angular-Applikation an einen Webserver}
% < Angular - Das umfassende Handbuch von: Christoph Höller
Um eine Anbindung an einen Webserver zu ermöglichen, stellt Angular ein Modul, welches den Namen HttpClientModule trägt, zur Verfügung. Nachdem dieses in die Klasse AppModule importiert wurde, kann es verwendet werden.
 
Wenn nun zum Beispiel die Daten eines GET-Requests aus dem Backend ausgelesen werden sollen, muss zuerst der HttpClient aufgerufen werden, definieren, ob ein GET, POST oder ein anderer Request benutzt werden soll und diesem muss anschließend eine passende URL übergeben werden. \ref{lst:impl:angularHTTPBsp}
 
\begin{lstlisting}[language=java,caption=Beispiel für einen GET-Request,label=lst:impl:angularHTTPBsp]
  this.http.get<Animal[]>(`http://localhost:8080/api/animals`)
    .subscribe(animals => {
      this.animalArray = animals;
    });
\end{lstlisting}
 
Mit den spitzen Klammern wird dabei der Type der übergebenen Daten definiert, in diesem Fall wäre es eine Liste des Typs Animal. Die erhaltenen Daten werden dann auf die Variable this.animalArray gesetzt. \cite{angularHandbuchBuch}
 
\subsubsection{Angular Forms}
% <  Angular - Das umfassende Handbuch von: Christoph Höller
Um in einem Angular-Projekt ein Formular verwenden zu können, muss zuerst die Forms-API hinzugefügt werden. Dies kann mit dem Befehl \texttt{npm install @angular/forms --save} umgesetzt werden. Nachdem die Module in der gewünschten Komponente eingebaut wurden, kann ein reaktives Formular erstellt werden.
 
Um auf die Eingabewerte eines Formulars zugreifen zu können, kann eine FormGroup verwendet werden. Bei dieser muss lediglich ein FormGroup-Element angelegt werden, welches die Namen der Eingabefelder trägt. \ref{lst:impl:angularFormGroup} Der linke Parameter ist dabei dazu da, einen Default-Wert festzulegen, der rechte ist dazu da, eine Überprüfung der Eingabewerte zu aktivieren.
 
\begin{lstlisting}[language=java,caption=Beispiel für FormGroup eines Angular Formulars,label=lst:impl:angularFormGroup]
  this.animalForm = this.fb.group({
    animalName: [null, [Validators.required]],
    animalAge: [0, [Validators.required]],
  });
\end{lstlisting}
 
Das zugehörige HTML-Form hat in der Praxis diese Form:  \ref{lst:impl:angularFormHTML}
 
 
\begin{lstlisting}[language=java,caption=Beispiel für ein reaktives Formular,label=lst:impl:angularFormHTML]
  <form [formGroup]="animalForm">
        <label>Tier Name</label>
        <input formControlName="animalName" type="text">
        <label>Tier Alter</label>
        <input formControlName="animalAge" type="number">
      </form>
\end{lstlisting}
\cite{angularHandbuchBuch}
 
 
\subsection{Bootstrap}
% https://getbootstrap.com/
Mithilfe von Bootstrap können schnelle und responsive Seiten gebaut werden. Bootstrap ist leistungsstart, erweiterbar und mit vielen Features versehen. Bootstrap stellt vielzählige Variablen für Farben, Schriftstile und mehr auf :root-Ebene zur Verfügung, welche überall verwendet werden können. Bei Komponenten sind CSS-Variablen auf die jeweilige Klasse beschränkt, können aber leicht geändert werden. 
\cite{bootstrap}

\subsection{CanvasJS}
% < 1 https://canvasjs.com/
 
CanvasJS ist ein Framework, mit welchem sich einfach Diagramme erstellen lassen können. Es besitzt eine hervorragende Integration in Angular, dies wird bestätigt durch eine eigene Unterseite auf der CanvasJS-Seite, welche der Integration mit Angular gewidmet ist. Auf dieser können unterschiedlichste Diagramme mit dem passenden Komponenten-, Modul- sowie HTML-Code gefunden werden.
 
Um CanvasChart in Angular zu verwenden, müssen verschiedene Codeteile eingefügt werden. Diese beinhalten einen HTML-Teil, welcher das Chart anzeigt. \ref{lst:impl:canvasJSHTML}
 
\begin{lstlisting}[language=html,caption=CanvasJS HTML,label=lst:impl:canvasJSHTML]
  <canvasjs-chart [options]="chartOptions" [styles]="{width: '100%', height: '360px'}"></canvasjs-chart>
\end{lstlisting}
 
Anschließend müssen Dateien, welche auf der canvasJS-Seite heruntergeladen werden können, in die Dateiordner des gewünschten Angular-Projekts kopiert werden. Diese tragen die Namen \texttt{canvasjs.min.js} und \texttt{canvasjs.angular.component.ts}.
 
Der vorletzte Schritt beinhaltet das Importieren des CanvasJSChart-Moduls.
Im letzten Schritt werden im TypeScript-File der Komponente, in der das Diagramm angezeigt werden soll, die Diagrammoptionen für das Diagramm festgelegt. Diese beinhalten in der einfachsten Form den Titel und die Daten, dies kann aber erweitert werden. \ref{lst:impl:canvasJSchartOptions}
 
\begin{lstlisting}[language=html,caption=CanvasJS chartOptiones,label=lst:impl:canvasJSchartOptions]
  chartOptions = {
    title: {
      text: "Basic Column Chart in Angular"
    },
    data: [{
      type: "line",
      dataPoints: [
        { label: "Apple",  y: 10  },
        { label: "Orange", y: 15  }
      ]}]};
\end{lstlisting}
 
\cite{canvasJsOfficialSite}