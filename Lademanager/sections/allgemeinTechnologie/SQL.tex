
% < 1 Einführung in SQL- Daten erzeugen, bearbeiten und abfragen von: Alan Beaulieu p15-27
% < 1 https://edufs.edu.htl-leonding.ac.at/moodle/pluginfile.php/102732/mod_resource/content/1/PLSQL-Skript%20v04.pdf
SQL ist eine Programmiersprache, welche entwickelt wurde, um die Daten in einer relationalen Datenbank zu bearbeiten. Außerdem kann damit die Struktur der Datenbank abgeändert werden. \cite{einfuerhungSQLBuch} 
 
Es gibt verschiedenste SQL-Befehlsgruppen, welche jeweils einen anderen Zweck erfüllen, siehe Abb. \ref{tab:allgemein:grundlagenSQL}.
Mithilfe all dieser verschiedenen Gruppen ist es möglich, eine Datenbank zu erstellen, zu verwalten und zu löschen.

\begin{table}
    \centering
    \begin{tabular}{ |c|c|c|c| }
     \hline
     \multicolumn{3}{|c|}{SQL-Befehlsgruppen } \\
     \hline
     \hline
     DB Manipulation (DML) & DB Definition (DDL) & DB Operation  \\
     \hline
     \hline
     DELETE & CREATE & SERVERERROR \\
     \hline
     INSERT & ALTER & LOGON/LOGOFF \\
     \hline
     UPDATE & DROP & STARTUP/SHUTDOWN \\
     \hline
    \end{tabular}
    \caption{Grundlagen SQL \cite{grundlagenSQL}}
    
    \label{tab:allgemein:grundlagenSQL}
\end{table}
 

 
Um beispielsweise eine Tabelle in einer Datenbank zu erstellen, wird das Schlüsselwort \texttt{CREATE} verwendet, welches in der Data Definition Language (DDL) gefunden werden kann. \ref{lst:impl:createTable}
 
\begin{lstlisting}[language=sql,caption=CREATE table,label=lst:impl:createTable]
    CREATE TABLE animal (
        animalName VARCHAR(20),
        animalAge NUMBER(3)
        );
\end{lstlisting}
 
Um nun eine Zeile in die Tabelle hinzuzufügen, wird ein INSERT-Statement benötigt. \ref{lst:impl:insertTable}
 
\begin{lstlisting}[language=sql,caption=INSERT INTO table,label=lst:impl:insertTable]
        INSERT INTO animal (animalName, animalAge) VALUES ('snake', 3);
\end{lstlisting}
 
Ein anderer wichtiger Befehl in SQL ist der SELECT-Befehl. Mit ihm kann aus jeder beliebigen Tabelle ein bestimmter Teil ausgegeben werden. Dabei können verschiedenste Kriterien übergeben werden, wie z.B. dass die auszugebenden Daten einen bestimmten Wert haben müssen. \cite{grundlagenSQL}
 
 
\subsection{Kommunikation mit Datenbanken}
% < 1 Einführung in SQL - Daten erzeugen, bearbeiten und abfragen von: Alan Beaulieu p29-47
Um SQL in einer Datenbank zu verwenden, muss dazu erst eine Datenbank angelegt werden. Dabei gibt es verschiedenste Datenbanken, welche je nach Performance, bzw. benötigter Größe ausgewählt werden sollte. Siehe dazu \ref{section:database}.
 
\subsubsection{Daten aus einer Datenbank auslesen}
Um mit den Daten einer Datenbank arbeiten zu können, müssen diese zuerst aus der Datenbank geholt werden. Dies passiert über ein SQL-Statement, welches ausgeführt werden muss. Dabei können einzelne Zeilen aus der Datenbank ausgewählt werden, oder auch mehrere Zeilen sowie ganze Tabelle. Es können verschiedene Kriterien in das SQL-Statement hinzugefügt werden, um unterschiedliche Ergebnisse zu erhalten, siehe Tab. \ref{tab:allgemein:selects}. \cite{einfuerhungSQLBuch}

\begin{table}
    \centering
    \begin{tabular}{ |c|c| }
     \hline
     \multicolumn{2}{|c|}{SELECT * FROM students } \\
     \hline
     \hline
     WHERE age > 18 & Alter muss über 18 liegen \\
     \hline
     \hline
     WHERE name == \glq Maria\grq{} & Name muss Maria betragen  \\
     \hline
     INSERT & ALTER \\
     \hline
     UPDATE & DROP  \\
     \hline
    \end{tabular}
    \caption{SQL-Select-Statements \cite{einfuerhungSQLBuch}}
    \label{tab:allgemein:selects}
\end{table}

    


\subsection{SQL und Java in Verbindung}

SQL und Java lassen sich verbinden. Durch eine Verbindung auf eine Datenbank mithilfe eines Connection-Blocks \ref{lst:impl:connect} können leicht Daten aus der Datenbank bearbeitet, Daten ausgelesen und Daten hinzugefügt werden. Dies wird mithilfe eines Strings, welches ein SQL-Statement enthält, umgesetzt  \ref{lst:impl:SQLString} und kurz darauf ausgeführt. 

\begin{lstlisting}[language=java,caption=SQL-String,label=lst:impl:SQLString]
    String sql = "SELECT * FROM flexlogger where timestamp > ? AND timestamp < ? order by timestamp";
\end{lstlisting}

Anschließend wird dieses in ein \texttt{PreparedStatement} umgewandelt und ausgeführt. Das Resultat daraus ist ein ResultSet. \ref{lst:impl:SQLStatement}

\begin{lstlisting}[language=java,caption=SQL-Statement ausführen,label=lst:impl:SQLStatement]
    try (Connection conn = connect()) {
        try (PreparedStatement ps = conn.prepareStatement(sql)) {
            ResultSet rs = ps.executeQuery();
    }}
\end{lstlisting}
