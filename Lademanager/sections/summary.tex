\setauthor{Teresa Holzer | Marcel Pouget}

Im Jahre 2021 wurden wir durch ein Ausschreiben der Firma FlexSolution auf das Unternehmen aufmerksam. Nach einem erfolgreichen Bewerbungsgespräch fing Marcel in der Firma als Werkstudent an und bekam die Möglichkeit, die Strukturen der Firma kennenzulernen. Nach 4 Monaten bekamen wir als Team dann das Angebot, im Sommer 2022 in der Firma ein Praktikum mit der Diplomarbeit zu verbinden. In der Arbeit sollte es vor allem um Energiemanagement gehen, konkreter mit Überwachung und Ansteuerung von verschiedenen Gerätschaften über die sieben Netzwerkschichten. Teresa wurde dann für die Arbeit an dem Logger zugeteilt, während Marcel für die Wallboxen zuständig war. Wir beschäftigten uns vor allem mit der Frage, wie man auf verschiedenen Netzwerkschichten Daten versenden, abgreifen, abspeichern und visuell aufbereiten kann.




Ausgangspunkt der Arbeit war, dass wir uns damit befassten, wie Daten übertragen werden. Danach teilte sich unser Aufgabengebiet auf. Während Marcel sich auf Modbus und serielle Übertragung fokussierte, war Teresa damit beschäftigt, einen Logger für extrem schnelle Datenströme zu planen. Sie konnte im Laufe des Projektes viel über Multithreading und die Abspeicherung von Daten lernen. Auch das Visualisieren auf einer Angular-Seite war ein großer Teil ihrer Arbeit, mit welchem sie sich in den 4 Wochen des Praktikums beschäftigt hat. Ihr Anliegen war es, Datenströme, welche durch Sensoren in der Firma entstehen, visuell und in zeitlicher Veränderung darzustellen. Das erlaubte dem Unternehmen, mögliche Fehlverhalten von Sensoren und Maschinen anhand der geloggten Daten zu erkennen.



Marcel war vor allem für die Ansteuerung der Ladestationen, der sogenannten Wallboxen zuständig. So war er verantwortlich dafür, dass die Datenübertragung auf serieller Ebene zuverlässig funktionierte und die Werte auf einer Weboberfläche angezeigt werden können. Er befasst sich vor allem mit Recherchen rund um Modbus TCP und Modbus RTU. Auch die Ansteuerung der PV-Anlagen fällt unter diesen Bereich.


In weiterer Folge übernahm der Kunde nach Abschluss des Projektes bereits erfolgreich die von uns entwickelte Software und integrierte sie erfolgreich in seine Systeme.


Der größte Erfolg des Projektteams war das Schaffen neuer Software, welche zuverlässig läuft und den Workflow der Firma verbessern kann. Probleme oder Rückschläge gab es kaum, denn dank der Hilfe vieler kompetenter Mitarbeiter wurde uns immer weitergeholfen, falls mal ein unüberwindbares Problem aufgetaucht ist.


Zusammenfassend kann man sagen, dass das Projekt "Elektroauto Lademanager" ein voller Erfolg war. Die Ziele unseres Projektes wurden alle in der geplanten Zeit eingehalten und unser Auftraggeber, Herr Pimminger, war mit der Arbeit des Teams sehr zufrieden. Um noch Änderungen oder andere Wünsche umsetzen zu können, stehen wir immer noch mit der Firma im engen Kontakt, um mögliche Fehler und Bugs zu lösen.