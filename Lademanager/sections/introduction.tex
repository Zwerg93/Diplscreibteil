\section{Herangehensweise}
\setauthor{Marcel Pouget}


\section{Zeitplan}
\setauthor{Marcel Pouget | Teresa Holzer}


\section{Verwendete Tools}



\subsection{Latex}
\subsection{Intellij}

\subsection{Webstorm}
%https://www.jetbrains.com/de-de/webstorm/
\setauthor{Teresa Holzer}
WebStorm wird als Entwicklungsumgebung für JavaScript und JavaScript-ähnlichen Sprachen definiert. Es wurde von JetBrains entwickelt und liefert eine Reihe von Hilfestellungen um JavaScript optimal programmieren zu können. WebStorm beinhaltet eine automatische Code-Überprüfung sowie die automatische Komplettierung von einzelnen Codezeilen. WebStrom beinhaltet außerdem einen bereits eingebauten Run- sowie Debug-Button, durch welchen das Starten in der Konsole wegfällt. Zusätzlich können in WebStorm verschiedenste Plugins hinzufügt werden, welche das Erscheinungsbild verändern, andere Sprachen überprüfen können und vieles mehr.     

\subsection{Filezilla}
\subsection{Linux Terminal}
\subsection{Discord}
\subsection{Google drive}
% https://one.google.com/about
Google Drive ist ein Cloud-Speicher, welcher für das Online-Speichern von Google Docs, Google Tabellen, Google Präsentationen und Google Formularen verwendet werden kann. Außerdem können Bild- und andere Formate hochgeladen und somit gesichert werden. Google Drive kann kostenlos durch das Erstellen eines Google-Accounts verwendet werden. Google stellt dabei 15GB Speicher frei zur Verfügung, danach kann durch en monatliches Abo-Modell mehr Speicherplatz erlangt werden.  

\subsection{vs code}

\subsection{Github}
% https://github.com/features
GitHub ist ein Online-Tool um zusammen mit Teammitgliedern einfach an Projekten zusammenzuarbeiten. Projekte können dabei einfach neu herunter geladen werden, es kann auf neue Versionen aktualisiert werden, und es können Versionen mit verschiedenen Versionsnummern zusammengefügt werden. GitHub kann über das Terminal oder GitHub Desktop verwendet werden. 
